\part*{Modelo Conceptual}

\section{Modelado del problema}

\subsection{Variables de control}
Se definen como variables de control del problema, las siguientes:

\begin{itemize}
    \item Cantidad de programadores que se ``reservan'' (para proyectos \textit{atractivos}).
    \item Estrategia de decisión de aceptación de proyectos.
\end{itemize}

\section{Variables aleatorias}
Se encontraron las siguientes variables aleatorias:

\begin{itemize}
    \item Cantidad de proyectos que llegan en un determinado período.
    \item Tipo del proyecto (pequeño, mediano o grande).
    \item Tamaño del proyecto (medido en horas-hombre).
    \item Precio por hora del proyecto.
    \item Fecha de entrega del proyecto.    
\end{itemize}

Los proyectos son modelados como una tupla de horas-hombre, precio por hora y fecha de entrega. La función de decisión sobre proyectos es la encargada de decidir 
si un proyecto es elegible o no.

\section{Plan de cuadros}
% TODO
TODO

\section{Funciones objectivo}
Las funciones objectivo que se consideran son:

\begin{itemize}
    \item Costo de oportunidad, ingreso que hubieran generado los proyectos rechazados.
    \item Ingreso generado por los proyectos aceptados.
    \item Porcentaje de recursos utilizados.
\end{itemize}

Para representar los resultados y la evolución de estas magnitudes, se tiene una vista que muestra, paso por paso, estos valores y, al final de la simulación, un
gráfico con los intervalos de valores. \\

Cabe destacar que para esto, previamente se debe debe fijar, para las estrategias de decisión que los tengan, los parámetros de entrada, buscando que sean los
que arrojen los mejores resultados.\\

\subsection{Restricciones}

Para el modelado del problema, se ignoraron varias variables reales del problema, que hubieran hecho impracticable la representación y resolución del mismo. El simulador se ve 
restringido a un uso académico, dado que no tiene en cuenta muchas variables del mundo real.

Las variables que se ignoran son: \\

\begin{itemize}
    \item La capacitación del personal y su curva de aprendizaje: se supone que el programador conoce el proyecto y no tarda en empezar a programar.
    \item Nuevos requerimientos en el proyecto: Desde un primer momento, se conoce el tiempo de desarrollo que involucrará un proyecto, no se pueden agregar 
            funcionalidades a la mitad del desarrollo.
    \item Competencia, mercado, precios: estas variables están sujetas a la llegada de proyectos, pero no se toman como variables separadas sino que se engloban en la 
        llegada de proyectos.
    \item La situación financiera de la empresa: se considera que la empresa tiene fondos como para pagar sus costos de funcionamiento durante la duración de la simulación.
    \item El trapaso de desarrolladores: se supone que la cantidad de desarrolladores permanece constante a lo largo de la simulación.
\end{itemize}


\section{Diagrama de Bloques}
TODO

\section{Diagrama de Flujo}

\begin{multicols}{2}
Muestras artificiales (MA)
Variables

\begin{enumerate*}
    \item Cantidad de Proyectos por mes(CA)\\
            VA Poisson ($\lambda$)
    \item Tipo de Proyecto (TP)\\
            VA Uniforme
    \item Tamaño de Proyecto [hs]\\
            VA Triangular(a, b, c)
    \item Costo Hora-Hombre [\$]\\
            VA Triangular(a, b, c)
    \item Plazo de Entrega del Proyecto [hs]\\
            VA Triangular(a, b, c)
\end{enumerate*}
\vfill
\columnbreak
\begin{itemize*}
 \item T: Tiempo del reloj de la simulación
 \item Tmax: Cantidad de tiempo a simular
 \item P: Proyecto actual
 \item R: Recursos (programadores)
 \item M: Dinero
 \item CO: Costo de Oportunidad
 \item M(P): Ganancia del proyecto P
 \item R(P): Función de asignación de recursos usados por el proyecto P
 \item T(P): Tiempo que abarca el proyecto P
 \item D: Función o Estrategia de decisión, toma un proyecto y devuelve si se lo acepta o no
 \item O(P\_1, ..., P\_n, D): Ordena los proyectos P\_1 a P\_n según el criterio de la estrategia D

\end{itemize*}

\end{multicols}


\section{Estrategias de decisión}

Tanto en el diagrama de bloques como en el diagrama de flujo se observa una función o estrategia de decisión para aceptar o no un proyecto. A continuación se detallan 
las estrategias que se estudiarán. Debe ser tomado en cuenta que independientemente de la estrategia, si no se tienen suficientes recursos como para completar el proyecto 
antes de su fecha de entrega el valor de $D(P)$, es decir la función que acepta o no el proyecto, es siempre $NO$.

\begin{enumerate}
    \item Se ordenan los proyectos arribados según el cociente Ganancia / Tamaño, y se aceptan los proyectos que tengan los mayores valores de tal cociente,
 o que obtengan valores mayores a un cierto umbral.
    Variables: $\mu$: valor mínimo de aceptación para el ratio Ganancia / Tamaño de un proyecto

    \item Se ordena los proyectos encolados según su tamaño, y se aceptan los de menor tamaño o aquellos que sean menor a un cierto tamaño. 
    Variables: $\tau$: valor máximo de tamaño de un proyecto para su aceptación

\end{enumerate}

\section{Estrategia de asignación de recursos}

Para la asignación de recursos, al comienzo de cada período, se calcula la cantidad de horas/hombre que se van a asignar al proyecto de la siguiente manera: 
se divide el total de horas/hombre restantes del proyecto, llamémosle W, y se lo divide por la 
cantidad de períodos que faltan hasta la fecha entrega. En caso de queden recursos sin asignar, se asignarán al proyecto más cercano a su fecha de entrega. 
De esta forma, se busca balancear la asignación de recursos, buscando cumplir con las fechas de entrega, y en caso de ser posible, adelantar los proyectos 
cercanos a su fin, con el objetivo de poder aceptar otros nuevos.
