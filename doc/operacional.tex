\part*{Modelo Operacional}

\section{Arquitectura}

Para el desarrollo del simulador se utilizará el lenguaje \textit{Python}. Se hará uso de la librería \textit{numpy} para la generación de números aleatorios, \textit{matplotlib}
para la generación de gráficos y \textit{wxPython} para la GUI.\\

\section{Etapas y Plan de Desarrollo}

\subsection*{Etapa 1 - Modelo de Organización}
Como primer paso del desarrollo, se hará el modelo de organización y proyectos. Esto es, codificar las reglas de decisión y asignación de recursos. Desde luego que para poder 
desarrollar estas reglas, se debe codificar los tipos de datos proyeto y organización.\\

\subsection*{Etapa 2 - Motor de Simulación}
El principal objetivo de esta etapa es poder simular el análisis mensual de la organización. Con la llegada de proyectos, la decisión de toma de proyectos y la asignación de los 
recursos para el mes de trabajo.\\

\subsection*{Etapa 3 - Modelo de Datos}
El objetivo de esta etapa es ajustar de forma correcta el valor de todas las variables y parámetros especificados en el modelo de datos. Se deberían implementar los 
generadores de cada variable y deberían ser verificados.\\

\subsection*{Etapa 4 - Interfaz visual}
El objetivo de esta etapa es que pueda visualizarse la corrida de la simulación, mostrando las diferentes etapas, procesos y/o resultados intermedios. También se deberían
poder visualizar las métricas en especial aquellas que son funciones objetivo.\\


\section{Plan de experimentación}
Para probar el simulador se diseñarán diferentes estrategias de aceptación y asignación de recursos, y se variará la cantidad de desarrolladores. 
Se probarán incluso escenarios de irreales, como la estrategia trivial de aceptar todo proyecto en según su orden de llegada, sólo con el motivo de probar el simulador 
y verificar que no surgen problemas.\\


\section{Plan de pruebas}

\subsection{Verificación}


\begin{itemize}
    \item \textbf{Verificación de generadores aleatorios:} En el simulador se colocaron los histogramas de los valores que se generan durante la simulación. El objetivo 
        de esto es verificar que en la simulación los histogramas se correspondan con las distribuciones de cada variable.
    \item \textbf{Verificación de estrategias de aceptación:} Una de las cosas importantes a verificar es que el simulador elija los caminos acorde a la estrategia seleccionada. 
        Para ello se diseñaron dos pequeños conjuntos de testeo, compuestos por proyectos, generados correctamente según las distribuciones que corresponden, para los cuales, 
        después de un corto tiempo de simulación se conocen los resultados.

\end{itemize}

