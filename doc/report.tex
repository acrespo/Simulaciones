\documentclass[a4paper,10pt]{article}

\usepackage[utf8]{inputenc}
\usepackage{t1enc}

\usepackage[utf8]{inputenc}
\usepackage{t1enc}
\usepackage[spanish]{babel}
\usepackage[pdftex,usenames,dvipsnames]{color}
\usepackage[pdftex]{graphicx}
\usepackage{enumerate}
\usepackage{amsmath}
\usepackage{amsfonts}
\usepackage{amssymb}
\usepackage[table]{xcolor}
\usepackage[small,bf]{caption}
\usepackage{float}
\usepackage{subfig}
\usepackage{listings}
\usepackage{bm}
\usepackage{times}
\usepackage{verbatim}
\usepackage{moreverb}
\usepackage{fancyvrb}
% \usepackage{hyperref}
\usepackage{multirow}
\usepackage{multicol}
\usepackage{mdwlist}
\usepackage{url}
\usepackage{listings}
\lstset{breaklines=true}
\lstset{numbers=left, numberstyle=\scriptsize\ttfamily, numbersep=10pt, captionpos=b} 
\lstset{basicstyle=\small\ttfamily}
\lstset{framesep=4pt}

\begin{document}
\setcounter{secnumdepth}{5}
\setcounter{tocdepth}{5}

\begin{titlepage}
        \vfill
        \thispagestyle{empty}
        \begin{center}
                \includegraphics{./images/itba_logo.png}
                \vfill
                \Huge{Simulación de Sistemas}\\
                \vspace{1cm}
                \Huge{Trabajo Pr\'actico Especial}\\
                \vspace{1cm}
                \Huge{Estrategia comercial de una software factory}\\
                \vspace{1cm}
                Grupo 3
        \end{center}
        \vfill
        \large{
        \begin{tabular}{lcr}
                Civile, Juan Pablo && 50453\\
                Crespo, Alvaro && 50758 \\
                Marseillan, Agustín && 50134\\
        \end{tabular}
}
        \vspace{2cm}
        \begin{center}
                \large{17 de Junio de 2013}\\
        \end{center}
\end{titlepage}
\newpage

\setcounter{page}{1}

\section{Introducción}

El presente Trabajo Práctico busca representar la estrategia comercial de una consultora de software (\textit{software factory}). En cuanto a la selección de proyectos, se busca 
la mejor combinación entre aprovechar el trabajo de los programadores y mantener cierta reserva de capacidad para no tener que rechazar proyectos \textit{atractivos}, es decir,
de mayor rentabilidad.\\

La simulación se basa en datos ``pseudo-reales'', dada la dificultad que implica la recolección de datos y la escasez de ellos.\\

\section{Modelo Conceptual}

\subsection{Modelado del problema}

\subsubsection{Variables de control}
Se definen como variables de control del problema, las siguientes:

\begin{itemize}
    \item Cantidad de programadores que se ``reservan'' (para proyectos \textit{atractivos}).
    \item Estrategia de decisión de aceptación de proyectos.
\end{itemize}

\subsection{Variables aleatorias}
Se encontraron las siguientes variables aleatorias:

\begin{itemize}
    \item Cantidad de proyectos que llegan en un determinado período.
    \item Tipo del proyecto (pequeño, mediano o grande).
    \item Tamaño del proyecto (medido en horas-hombre).
    \item Precio por hora del proyecto.
    \item Fecha de entrega del proyecto.    
\end{itemize}

Los proyectos son modelados como una tupla de horas-hombre, precio por hora y fecha de entrega. La función de decisión sobre proyectos es la encargada de decidir 
si un proyecto es elegible o no.

\subsection{Plan de cuadros}
% TODO
TODO

\subsection{Funciones objectivo}
Las funciones objectivo que se consideran son:

\begin{itemize}
    \item Costo de oportunidad, ingreso que hubieran generado los proyectos rechazados.
    \item Ingreso generado por los proyectos aceptados.
    \item Porcentaje de recursos utilizados.
\end{itemize}

Para representar los resultados y la evolución de estas magnitudes, se tiene una vista que muestra, paso por paso, estos valores y, al final de la simulación, un
gráfico con los intervalos de valores. \\

Cabe destacar que para esto, previamente se debe debe fijar, para las estrategias de decisión que los tengan, los parámetros de entrada, buscando que sean los
que arrojen los mejores resultados.\\

\subsubsection{Restricciones}

Para el modelado del problema, se ignoraron varias variables reales del problema, que hubieran hecho impracticable la representación y resolución del mismo. El simulador se ve 
restringido a un uso académico, dado que no tiene en cuenta muchas variables del mundo real.

Las variables que se ignoran son: \\

\begin{itemize}
    \item La capacitación del personal y su curva de aprendizaje: se supone que el programador conoce el proyecto y no tarda en empezar a programar.
    \item Nuevos requerimientos en el proyecto: Desde un primer momento, se conoce el tiempo de desarrollo que involucrará un proyecto, no se pueden agregar 
            funcionalidades a la mitad del desarrollo.
    \item Competencia, mercado, precios: estas variables están sujetas a la llegada de proyectos, pero no se toman como variables separadas sino que se engloban en la 
        llegada de proyectos.
    \item La situación financiera de la empresa: se considera que la empresa tiene fondos como para pagar sus costos de funcionamiento durante la duración de la simulación.
    \item El trapaso de desarrolladores: se supone que la cantidad de desarrolladores permanece constante a lo largo de la simulación.
\end{itemize}


\subsection{Diagrama de Bloques}
TODO

\clearpage
\subsection{Diagrama de Flujo}

\begin{multicols}{2}
Muestras artificiales (MA)
Variables

\begin{enumerate*}
    \item Cantidad de Proyectos por mes(CA)\\
            VA Poisson ($\lambda$)
    \item Tipo de Proyecto (TP)\\
            VA Uniforme
    \item Tamaño de Proyecto [hs]\\
            VA Triangular(a, b, c)
    \item Costo Hora-Hombre [\$]\\
            VA Triangular(a, b, c)
    \item Plazo de Entrega del Proyecto [hs]\\
            VA Triangular(a, b, c)
\end{enumerate*}
\vfill
\columnbreak
\begin{itemize*}
 \item T: Tiempo del reloj de la simulación
 \item Tmax: Cantidad de tiempo a simular
 \item P: Proyecto actual
 \item R: Recursos (programadores)
 \item M: Dinero
 \item CO: Costo de Oportunidad
 \item M(P): Ganancia del proyecto P
 \item R(P): Función de asignación de recursos usados por el proyecto P
 \item T(P): Tiempo que abarca el proyecto P
 \item D: Función o Estrategia de decisión, toma un proyecto y devuelve si se lo acepta o no
 \item O(P\_1, ..., P\_n, D): Ordena los proyectos P\_1 a P\_n según el criterio de la estrategia D

\end{itemize*}

\end{multicols}


\subsection{Estrategias de decisión}

Tanto en el diagrama de bloques como en el diagrama de flujo se observa una función o estrategia de decisión para aceptar o no un proyecto. A continuación se detallan 
las estrategias que se estudiarán. Debe ser tomado en cuenta que independientemente de la estrategia, si no se tienen suficientes recursos como para completar el proyecto 
antes de su fecha de entrega el valor de $D(P)$, es decir la función que acepta o no el proyecto, es siempre $NO$.

\begin{enumerate}
    \item Se ordenan los proyectos arribados según el cociente Ganancia / Tamaño, y se aceptan los proyectos que tengan los mayores valores de tal cociente,
 o que obtengan valores mayores a un cierto umbral.
    Variables: $\mu$: valor mínimo de aceptación para el ratio Ganancia / Tamaño de un proyecto

    \item Se ordena los proyectos encolados según su tamaño, y se aceptan los de menor tamaño o aquellos que sean menor a un cierto tamaño. 
    Variables: $\tau$: valor máximo de tamaño de un proyecto para su aceptación

\end{enumerate}

\subsection{Estrategia de asignación de recursos}

Para la asignación de recursos, al comienzo de cada período, se calcula la cantidad de horas/hombre que se van a asignar al proyecto de la siguiente manera: 
se divide el total de horas/hombre restantes del proyecto, llamémosle W, y se lo divide por la 
cantidad de períodos que faltan hasta la fecha entrega. En caso de queden recursos sin asignar, se asignarán al proyecto más cercano a su fecha de entrega. 
De esta forma, se busca balancear la asignación de recursos, buscando cumplir con las fechas de entrega, y en caso de ser posible, adelantar los proyectos 
cercanos a su fin, con el objetivo de poder aceptar otros nuevos.

\section{Modelo de Datos}

\subsection{Variables Aleatoria}

\begin{itemize}
    \item[A)] Cantidad de proyectos por mes
    \item[B)] Tipo de proyecto
    \item[C)] Tamaño de proyecto [h]
    \item[D)] Precio hora-hombre del proyecto [\$]
    \item[E)] Plazo de entrega del proyecto [h]
\end{itemize}

\subsection{Estrategia de obtención}
TODO


\textbf{Cuadro de valores para las distribuciones según tipo de proyecto}
\\

\begin{tabular}{|l|c|}
\hline
    Tipo        & Porcentaje\\
\hline
    Grande      & 15\%\\
\hline
    Mediano     & 75\%\\
\hline
    Pequeño     & 10\%\\
\hline
\end{tabular}
\\ 
\paragraph*{}


\begin{tabular}{|l|l|l|l|l|}
\hline
                    & Tipo      & Mín.    & Máx.    & Moda\\
\hline
    Horas           & Grande    & 4500    & 8000    & 5300\\
\hline
                    & Mediano   & 2000    & 4500    & 3200\\
\hline
                    & Pequeño   & 500     & 2000    & 3200\\
\hline
    Precio          & Grande    & 180     & 310     & 230\\
\hline
                    & Mediano   & 200     & 240     & 220\\
\hline
                    & Pequeño   & 200     & 300     & 290\\
\hline
    Fecha de Entrega& Grande    & Horas/6 & Horas/4 & Horas/5\\
\hline
                    & Mediano   & Horas/4 & Horas/2 & Horas/3\\
\hline
                    & Pequeño   & Horas/3 & Horas/1 & Horas/2\\
\hline\end{tabular}

\section{Modelo Operacional}

\subsection{Arquitectura}

Para el desarrollo del simulador se utilizará el lenguaje \textit{Python}. Se hará uso de la librería \textit{numpy} para la generación de números aleatorios, \textit{matplotlib}
para la generación de gráficos y \textit{wxPython} para la GUI.\\

\subsection{Etapas y Plan de Desarrollo}

\subsection*{Etapa 1 - Modelo de Organización}
Como primer paso del desarrollo, se hará el modelo de organización y proyectos. Esto es, codificar las reglas de decisión y asignación de recursos. Desde luego que para poder 
desarrollar estas reglas, se debe codificar los tipos de datos proyeto y organización.\\

\subsection*{Etapa 2 - Motor de Simulación}
El principal objetivo de esta etapa es poder simular el análisis mensual de la organización. Con la llegada de proyectos, la decisión de toma de proyectos y la asignación de los 
recursos para el mes de trabajo.\\

\subsection*{Etapa 3 - Modelo de Datos}
El objetivo de esta etapa es ajustar de forma correcta el valor de todas las variables y parámetros especificados en el modelo de datos. Se deberían implementar los 
generadores de cada variable y deberían ser verificados.\\

\subsection*{Etapa 4 - Interfaz visual}
El objetivo de esta etapa es que pueda visualizarse la corrida de la simulación, mostrando las diferentes etapas, procesos y/o resultados intermedios. También se deberían
poder visualizar las métricas en especial aquellas que son funciones objetivo.\\


\subsection{Plan de experimentación}
Para probar el simulador se diseñarán diferentes estrategias de aceptación y asignación de recursos, y se variará la cantidad de desarrolladores. 
Se probarán incluso escenarios de irreales, como la estrategia trivial de aceptar todo proyecto en según su orden de llegada, sólo con el motivo de probar el simulador 
y verificar que no surgen problemas.\\


\subsection{Plan de pruebas}

\subsubsection{Verificación}


\begin{itemize}
    \item \textbf{Verificación de generadores aleatorios:} En el simulador se colocaron los histogramas de los valores que se generan durante la simulación. El objetivo 
        de esto es verificar que en la simulación los histogramas se correspondan con las distribuciones de cada variable.
    \item \textbf{Verificación de estrategias de aceptación:} Una de las cosas importantes a verificar es que el simulador elija los caminos acorde a la estrategia seleccionada. 
        Para ello se diseñaron dos pequeños conjuntos de testeo, compuestos por proyectos, generados correctamente según las distribuciones que corresponden, para los cuales, 
        después de un corto tiempo de simulación se conocen los resultados.

\end{itemize}


\section{Conclusiones}


\clearpage
\appendix
\section{Anexo}

\end{document}
