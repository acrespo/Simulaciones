\documentclass[a4paper,10pt]{article}

\usepackage[utf8]{inputenc}
\usepackage{t1enc}

\usepackage[utf8]{inputenc}
\usepackage{t1enc}
\usepackage[spanish]{babel}
\usepackage[pdftex,usenames,dvipsnames]{color}
\usepackage[pdftex]{graphicx}
\usepackage{enumerate}
\usepackage{amsmath}
\usepackage{amsfonts}
\usepackage{amssymb}
\usepackage[table]{xcolor}
\usepackage[small,bf]{caption}
\usepackage{float}
\usepackage{subfig}
\usepackage{listings}
\usepackage{bm}
\usepackage{times}
\usepackage{verbatim}
\usepackage{moreverb}
\usepackage{fancyvrb}
% \usepackage{hyperref}
\usepackage{multirow}
\usepackage{url}
\usepackage{listings}
\lstset{breaklines=true}
\lstset{numbers=left, numberstyle=\scriptsize\ttfamily, numbersep=10pt, captionpos=b} 
\lstset{basicstyle=\small\ttfamily}
\lstset{framesep=4pt}

\begin{document}
\setcounter{secnumdepth}{5}
\setcounter{tocdepth}{5}

\begin{titlepage}
        \vfill
        \thispagestyle{empty}
        \begin{center}
                \includegraphics{./images/itba_logo.png}
                \vfill
                \Huge{Simulación de Sistemas}\\
                \vspace{1cm}
                \Huge{Trabajo Pr\'actico Especial}\\
                \vspace{1cm}
                \Huge{Estrategia comercial de una software factory}\\
                \vspace{1cm}
                Grupo 3
        \end{center}
        \vfill
        \large{
        \begin{tabular}{lcr}
                Civile, Juan Pablo && 50453\\
                Crespo, Alvaro && 50758 \\
                Marseillan, Agustín && 50134\\
        \end{tabular}
}
        \vspace{2cm}
        \begin{center}
                \large{17 de Junio de 2013}\\
        \end{center}
\end{titlepage}
\newpage

\setcounter{page}{1}

\section{Introducción}

El presente Trabajo Práctico busca representar la estrategia comercial de una consultora de software (\textit{software factory}). En cuanto a la selección de proyectos, se busca 
la mejor combinación entre aprovechar el trabajo de los programadores y mantener cierta reserva de capacidad para no tener que rechazar proyectos \textit{atractivos}, es decir,
de mayor rentabilidad.\\

La simulación se basa en datos ``pseudo-reales'', dada la dificultad que implica la recolección de datos y la escasez de ellos.\\

\subsection{Modelado del problema}

\subsubsection{Variables de control}
Se definen como variables de control del problema, las siguientes:

\begin{itemize}
    \item Cantidad de programadores que se ``reservan'' (para proyectos \textit{atractivos}).
    \item Estrategia de decisión de aceptación de proyectos.
\end{itemize}

\subsection{Variables aleatorias}
Se encontraron las siguientes variables aleatorias:

\begin{itemize}
    \item Cantidad de proyectos que llegan en un determinado período.
    \item Tipo del proyecto (pequeño, mediano o grande).
    \item Tamaño del proyecto (medido en horas-hombre).
    \item Precio por hora del proyecto.
    \item Fecha de entrega del proyecto.    
\end{itemize}

Los proyectos son modelados como una tupla de horas-hombre, precio por hora y fecha de entrega. La función de decisión sobre proyectos es la encargada de decidir 
si un proyecto es elegible o no.

\subsection{Plan de cuadros}
% TODO
TODO

\subsection{Funciones objectivo}
Las funciones objectivo que se consideran son:

\begin{itemize}
    \item Costo de oportunidad, ingreso que hubieran generado los proyectos rechazados.
    \item Ingreso generado por los proyectos aceptados.
    \item Porcentaje de recursos utilizados.
\end{itemize}

Para representar los resultados y la evolución de estas magnitudes, se tiene una vista que muestra, paso por paso, estos valores y, al final de la simulación, un
gráfico con los intervalos de valores. \\

Cabe destacar que para esto, previamente se debe debe fijar, para las estrategias de decisión que los tengan, los parámetros de entrada, buscando que sean los
que arrojen los mejores resultados.\\

\subsubsection{Restricciones}


\section{Desarrollo}


\section{Conclusiones}


\clearpage
\appendix
\section{Anexo}

\end{document}
